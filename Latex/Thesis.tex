\documentclass[12pt, oneside]{article}
\usepackage{amsmath}
\usepackage{datetime}
\usepackage{todonotes}
\usepackage{amsfonts}
\usepackage{subfiles}
\usepackage[toc, page]{appendix}
\usepackage{tikz,fullpage}
\usepackage[export]{adjustbox}
\usepackage{tikz}
\usepackage{footnote}
\usetikzlibrary{tikzmark}
\usepackage{amsthm}
\usepackage{listings}

\usepackage{adjustbox}
\usepackage[backend=biber, style=bibstyle/lncs]{biblatex}
\addbibresource{sources.bib}
\renewcommand{\familydefault}{\sfdefault}
\interfootnotelinepenalty=10000
\usetikzlibrary{arrows.meta}
\usetikzlibrary{arrows, decorations.markings}
\tikzstyle{vecArrow} = [thick, decoration={markings,mark=at position
   1 with {\arrow[semithick]{Triangle[open,length=3mm,width=5mm]}}},
   double distance=2.4pt, shorten >= 8pt,
   preaction = {decorate},
   postaction = {draw,line width=2.4pt, white,shorten >= 5pt}]
\theoremstyle{plain}
\newtheorem{thm}{Theorem}[section]
\theoremstyle{definition}
\newtheoremstyle{indented}{3pt}{3pt}{\addtolength{\leftskip}{2.5em}}{
}{\bfseries}{.}{.5em}{}

\theoremstyle{indented}
\newtheorem{defn}[thm]{Definition}

\title{FPGA-on-FPGA emulation}
\author{Pim van Leeuwen}
\date{\today}

\begin{document}

\maketitle



\begin{abstract}
FPGAs allow reconfiguration of its logic at any point after production. The result is that they are effective at prototyping application-specific integrated circuits, updating the internal logic while in the field and at low-cost low-quantity use cases. To optimise these processes, it is crucial to properly educate engineers in the implementation of FPGA programs and the FPGA compilation process. Traditional FPGA programming pipelines involve a computationally expensive (NP-hard) place \& route process that slows down iterations of FPGA programs and hinders the educational process. We propose a virtual environment in which place \& route is performed manually in which the student learns about the intricacies of place \& route and in which compilation is linear. To this end, we require emulation of a virtual FPGA on a physical, concrete FPGA. In the proposed research, we find such an emulation using an algorithm that solves a variant of subgraph isomorphism. This algorithm aims to find emulations in as many cases as possible and to exploit the hierarchy of FPGAs to speed up computation. The expected result is a software package that computes emulators which each output a program for a concrete FPGA provided with a program for a virtual FPGA.
\end{abstract}

\newpage
\tableofcontents
\section{Introduction}

\section{Background}
\subsection{Field Programmable Gate Arrays}
\subsubsection{Lookup tables}
\subsubsection{Registers}
\subsubsection{Logic Cells}
\subsubsection{Routing}
\subsubsection{Compilation}
\subsection{Simulation versus Emulation}
\subsection{Path subgraph isomorphism}
\section{Models}
\subsection{FPGA}
Our algorithm will generate an emulation using path subgraph isomorphism. To this end, we will model FPGA designs as graphs. For this purpose, let us define the type $hierarchygraph : (V, E, L, H)$ where $V$ is a set of vertices, $E\subseteq (V \times V)$ is a set of undirected edges, $L$ is a vertex labeling function $V \to \lambda$ where $\lambda$ is a finite set of labels and $H$ describes the hierarchy with a relation $V \to hierarchygraph$. We restrict ourselves to graphs without loops, i.e. $\not \exists v \in V . (v, v) \in E$. Let $model(B)$ be the $hierarchygraph$ model of (a subset of) an FPGA layout $B$. Then:

\begin{itemize}
\item Whenever $B$ has a connection to external hardware (e.g. a pin) with an optional identifier $k$ where order is relevant, $model(B)$ has an edge to a distinct vertex $v$ and $L(v)=``port\_out_k"$.

\item Each component that is one step lower than $B$ in the FPGA hierarchy (e.g. CLBs) corresponds to a distinct vertex $v$ such that $L(v)=``component"$ and with a set of vertices $S$ with optional identifiers such that $\forall s \in S . L(s)=``port\_in_{id(s)}" \land (s,v) \in E$. Furthermore, the contents of this component are recursively defined in $H(v)$.

\item Whenever $B$ has a routing switch, $model(B)$ has a distinct vertex $v$ and $L(v)=``switch"$.

\item Whenever $B$ has a mux with $n$ wires for each input and for its output, $model(B)$ has distinct vertices $\{v_{left, 1} \dots v_{left, n}\} \cup \{v_{right, 1} \dots v_{right, n}\} \cup \{v_{out, 1} \dots v_{out, n}\} \cup \{v_{select}\}$ where:

\begingroup
\setlength{\tabcolsep}{2pt} % Default value: 6pt
\begin{tabular}{llllll}
$\forall v_{left, i}$ & $\in \{v_{left, 1} \dots v_{left, n}\}$ & $. L(v_{left, i}$&$= ``mux\_left_i"$&$\land (v_{left, i}$&$, v_{select}) \in E$,\\
$\forall v_{right, i}$ & $\in \{v_{right, 1} \dots v_{right, n}\}$ & $. L(v_{right, i})$&$= ``mux\_right_i"$&$\land (v_{right, i}$&$, v_{select}) \in E$,\\
$\forall v_{out, i}$ & $\in \{v_{out, 1} \dots v_{out, n}\}$ & $. L(v_{out, i}) $&$= ``mux\_out_i"$&$\land (v_{out, i}$&$, v_{select}) \in E$ and
\end{tabular}
\endgroup

$L(v_{select})=``mux\_select"$

Whenever something is connected to input or output of the mux in $B$, $model(B)$ has an edge to the corresponding vertex $v \in \{v_{left, 1} \dots v_{left, n}\} \cup \{v_{right, 1} \dots v_{right, n}\} \cup \{v_{out, 1} \dots v_{out, n}\}$. Furthermore, if $B$ has a wire to the selection input, $model(B)$ has an edge to the vertex $v_{select}$.

\item Each LUT that has $n$ inputs and $m$ outputs corresponds to\todo{change to $model(B)$} unique vertices $v_{in}$ and $v_{out}$ such that $L(v_{in})=``lut_{in}"$, $L(v_{out})=``lut_{out}"$ and $(v_{in}, v_{out}) \in E$. Wires to the input of the LUT correspond with edges to $v_{in}$ whilst wires from the output of the LUT correspond with edges from $v_{out}$.

\item Each register that has $n$ inputs corresponds\todo{change to $model(B)$} to:

$\{v_{in, 1} \dots v_{in, n}\} \cup \{v_{out, 1} \dots v_{out, n}\} \cup \{v_{set}, v_{resetsync}, v_{resetasync}\}$ where

\begingroup
\setlength{\tabcolsep}{2pt} % Default value: 6pt
\begin{tabular}{llllll}
$\forall v_{in, i}$ & $\in \{v_{in, 1} \dots v_{in, n}\}$ & $. L(v_{in, i})$&$=``register\_in_i"$&$\land (v_{in, i}, v_{set})$&$\in E$, \\
$\forall v_{out, i}$ & $\in \{v_{out, 1} \dots v_{out, n}\}$ & $. L(v_{out, i})$&$=``register\_out_i"$&$\land (v_{out, i}, v_{set})$&$\in E$, \\
\end{tabular}
\endgroup

$L(v_{set})=``register\_set"$,

\begingroup
\setlength{\tabcolsep}{2pt} % Default value: 6pt
\begin{tabular}{llll}
$L(v_{resetsync})$&$=``register\_reset\_sync"$&$\land (v_{set}, v_{resetsync}) $&$\in E$ and\\
$L(v_{resetasync})$&$=``register\_resFet\_async"$&$\land (v_{set}, v_{resetasync}) $&$\in E$.
\end{tabular}
\endgroup

A connection to an input of the register at wire position $x$ corresponds with\todo{change to $model(B)$} an edge to $v_{in, x}$. Similarly, a connection to an output of the register at wire position $x$ corresponds with\todo{change to $model(B)$} an edge to $v_{out, x}$. Connections to the set-, synchronous reset and asynchronous reset gates in the FPGA correspond to\todo{change to $model(B)$} edges to the respective vertex in $\{v_{set}, v_{resetsync}, v_{resetasync}\}$.
 

\item Each connection by wire corresponds to\todo{change to $model(B)$} an edge.
\end{itemize}
\subsection{FPGA program}
%\section{Preliminaries}

\section{Algorithm}
\subsection{State space storage}
\subsection{State space exploration}
\subsection{Hierarchy usage}
\subsection{Defining $f$}
\section{Performance experiments}
\section{Software design}
\subsection{Architecture}
\subsection{Manual}
\section{Conclusion}
\section{Discussion}
\section{Future Research}


\begin{appendices}
\section{History of subgraph isomorphism algorithms}
\label{app:algorithmHistory}
\adjincludegraphics[width=\textwidth, trim={0 {.555\height} 0 0},clip]{images/algorithmhistory.png}
\adjincludegraphics[width=\textwidth, trim={0 0 0 {.445\height}},clip]{images/algorithmhistory.png}
\end{appendices}

\newpage
\printbibliography

\end{document}

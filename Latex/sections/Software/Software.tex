\section{Software design}
\subsection{Architecture}
\subsection{Formats}
\subsubsection{FPGA layout input}
When providing a hierarchygraph $G=(V, E, L, H, C)$ to our software package, you need to provide the graph $(V, E, L)$ in a file in a GraphML\cite{Brandes04graphmarkup} format. This file starts with a prefix:

\lstset{
  language=XML,
  morekeywords={encoding, graphml, graph, node, edge, source, target, key, data, id}
}
\begin{lstlisting}[frame=single]
<?xml version="1.0" encoding="UTF-8"?>
<graphml xmlns="http://graphml.graphdrawing.org/xmlns"  
    xmlns:xsi="http://www.w3.org/2001/XMLSchema-instance"
    xsi:schemaLocation=
    "http://graphml.graphdrawing.org/xmlns/1.0/graphml.xsd">
  <key id="label"
    for="node"
    attr.name="label"
    attr.type="string"/>
  <key id="graphref"
    for="node"
    attr.name="is hierarchygraph"
    attr.type="string"/>
  <key id="noderef"
    for="node"
    attr.name="linked to"
    attr.type="string"/>
\end{lstlisting}


Next, you have to create a \texttt{graph} XML element:

\begin{lstlisting}[frame=single]
<graph id="myfpga" edgedefault="undirected">
\end{lstlisting}


where \texttt{myfpga} should be replaced by any sensible identifier for this hierarchygraph. Next, for each vertex $v$ in $V$ with $L(v)=\{label_1 \dots label_n\}$ you must add a \texttt{node} XML-element within the \texttt{graph} XML element:


\begin{lstlisting}[mathescape, frame=single]
<node id="myNode">
  <data key="label">$label_1$</data>
  $\dots$
  <data key="label">$label_n$</data>
</node>
\end{lstlisting}

Where \texttt{myNode} should be replaced by any sensible identifier for this node. Next, each ``component" node $v \in V, L(v)=\{``component"\}$ should have a reference to a file with hierarchygraph $H(v)$: 

\begin{lstlisting}[mathescape, frame=single]
<node id="myComponentNode">
  <data key="label">component</data>
  <data key="graphref">myComponent.graphml</data>
</node>
\end{lstlisting}

This file should be in the same directory as the file referencing to it. Next, each ``port" node $v \in V, L(v)=\{``port"\}$ should have a reference to a node identifier from another hierarchygraph:

\begin{lstlisting}[mathescape, frame=single]
<node id="myPortNode">
  <data key="label">port</data>
  <data key="noderef">referredNode</data>
</node>
\end{lstlisting}

where \texttt{referredNode} is the \texttt{id} of a node in the hierarchygraph that the ``component" node of this ``port" node references.

Next, for each edge $(v_1, v_2) \in E$ you must add an edge XML-element within the \texttt{graph} XML-element:

\begin{lstlisting}[frame=single, mathescape]
<edge id="myEdge" source="$id(v_1)$" target="$id(v_2)$"/>
\end{lstlisting}

where \texttt{myEdge} is replaced by a sensible identifier for this edge and $id(v_1)$ and $id(v_2)$ are the identifiers assigned to the nodes of this edge (i.e. the value of the id-attribute for \texttt{node} XML-elements). Lastly, you must close the \texttt{graph}- and \texttt{graphml} XML-elements to get a valid hierarchygraph document.

\begin{lstlisting}[frame=single]
</graph>
</graphml>
\end{lstlisting}

A complete example of the input format of a simple FPGA hierarchygraph (shown in Figure \ref{fig:example:simplemux}) is shown in Figures \ref{fig:example:simplemuxListing1} and \ref{fig:example:simplemuxListing2}.

\tikzstyle{graphnode} = [circle, minimum width=1.5cm, minimum height=1.5cm,text centered, draw=black]
\begin{figure}
\centering
\begin{adjustbox}{width=	\textwidth}
\begin{tikzpicture}
\draw (0,0) node[graphnode] (pin1) {pin};
\draw (2,0) node[graphnode] (port1) {port};
\draw (0,2) node[graphnode] (pin2) {pin};
\draw (2,2) node[graphnode] (port2) {port};
\draw (8,0) node[graphnode] (pin3) {pin};
\draw (6,0) node[graphnode] (port3) {port};
\draw (8,2) node[graphnode] (pin4) {pin};
\draw (6,2) node[graphnode] (port4) {port};
\draw (4,1) node[graphnode] (component) {component};
\draw (pin1)--(port1);
\draw (pin2)--(port2);
\draw (pin3)--(port3);
\draw (pin4)--(port4);
\draw (component)--(port1);
\draw (component)--(port2);
\draw (component)--(port3);
\draw (component)--(port4);

\draw(12, 1)  node[graphnode] (select) {select};
\draw(12, -1) node[graphnode] (in1)    {in};
\draw(12, 3)  node[graphnode] (in2)    {in};
\draw(14, -1) node[graphnode] (mux1)   {mux};
\draw(14, 3)  node[graphnode] (mux2)   {mux};
\draw(16, 1)  node[graphnode] (out)    {out};

\draw (in1)--(mux1);
\draw (in2)--(mux2);
\draw (in1)--(out);
\draw (in2)--(out);
\draw (select)--(mux1);
\draw (select)--(mux2);

\draw (10, 4)--(10,-2);
\draw[dotted] (port2)..controls(3,3)..(in2);
\draw[dotted] (port4)..controls(6,1)..(select);
\draw[dotted] (port3)..controls(6,-1)..(in1);
\draw[dotted] (port1)..controls(2,-3) and (12,0)..(out);
\end{tikzpicture}
\end{adjustbox}
\caption{An FPGA hierarchygraph with four pins and a 1-wire mux.}
\label{fig:example:simplemux}
\end{figure}

\lstset{title={virtualFPGA.graphml}}
\begin{figure}[h]
\centering
\begin{lstlisting}[frame=single,  basicstyle=\scriptsize]
<?xml version="1.0" encoding="UTF-8"?>
<graphml xmlns="http://graphml.graphdrawing.org/xmlns"  
    xmlns:xsi="http://www.w3.org/2001/XMLSchema-instance"
    xsi:schemaLocation=
    "http://graphml.graphdrawing.org/xmlns/1.0/graphml.xsd">
  <key id="label"
    for="node"
    attr.name="label"
    attr.type="string"/>
  <key id="graphref"
    for="node"
    attr.name="is hierarchygraph"
    attr.type="string"/>
  <key id="noderef"
    for="node"
    attr.name="linked to"
    attr.type="string"/>
  <graph id="virtualFPGA" edgedefault="undirected">
    <node id="n0"><data key="label">pin</data></node>
    <node id="n1"><data key="label">pin</data></node>
    <node id="n2"><data key="label">pin</data></node>
    <node id="n3"><data key="label">pin</data></node>
    <node id="n4">
      <data key="label">component</data>
      <data key="graphref">mymux.graphml</data>
    </node>
    <node id="n5">
      <data key="label">port</data>
      <data key="noderef">in1</data>
    </node>
    <node id="n6">
      <data key="label">port</data>
      <data key="noderef">in2</data>
    </node>
    <node id="n7">
      <data key="label">port</data>
      <data key="noderef">select</data>
    </node>
    <node id="n8">
      <data key="label">port</data>
      <data key="noderef">out</data>
    </node>
    <edge id="e0" source="n0" target="n5"/>
    <edge id="e1" source="n1" target="n6"/>
    <edge id="e2" source="n2" target="n7"/>
    <edge id="e3" source="n3" target="n8"/>
    <edge id="e4" source="n4" target="n5"/>
    <edge id="e5" source="n4" target="n6"/>
    <edge id="e6" source="n4" target="n7"/>
    <edge id="e7" source="n4" target="n8"/>
  </graph>
</graphml>
\end{lstlisting}
\caption{Hierarchygraph input of the left hierarchygraph in Figure \ref{fig:example:simplemux}}
\label{fig:example:simplemuxListing1}
\end{figure}

\lstset{title={mymux.graphml}}
\begin{figure}[h]
\centering
\begin{lstlisting}[frame=single,  basicstyle=\scriptsize]
<?xml version="1.0" encoding="UTF-8"?>
<graphml xmlns="http://graphml.graphdrawing.org/xmlns"  
    xmlns:xsi="http://www.w3.org/2001/XMLSchema-instance"
    xsi:schemaLocation=
    "http://graphml.graphdrawing.org/xmlns/1.0/graphml.xsd">
  <key id="label"
    for="node"
    attr.name="label"
    attr.type="string"/>
  <key id="graphref"
    for="node"
    attr.name="is hierarchygraph"
    attr.type="string"/>
  <key id="noderef"
    for="node"
    attr.name="linked to"
    attr.type="string"/>
  <graph id="1-wire mux" edgedefault="undirected">
    <node id="in1"><data key="label">in</data></node>
    <node id="in2"><data key="label">in</data></node>
    <node id="select"><data key="label">select</data></node>
    <node id="mux1"><data key="label">mux</data></node>
    <node id="mux2"><data key="label">mux</data></node>
    <node id="out"><data key="label">out</data></node>
   
    <edge id="e0" source="in1" target="out"/>
    <edge id="e1" source="in1" target="mux1"/>
    <edge id="e2" source="in2" target="out"/>
    <edge id="e3" source="in2" target="mux2"/>
    <edge id="e4" source="select" target="mux1"/>
    <edge id="e5" source="select" target="mux2"/>
  </graph>
</graphml>
\end{lstlisting}
\caption{Hierarchygraph input of the right hierarchygraph in Figure \ref{fig:example:simplemux}}
\label{fig:example:simplemuxListing2}
\end{figure}
\subsubsection{FPGA program input}
\subsubsection{FPGA program output}

\subsection{Manual}
\chapter{Pruning}
\label{chapter:pruning}

During the search for a complete matching, during which the algorithm only has \textit{partial} matchings, the algorithm will often explore dead branches, i.e. branches of the search tree that will not eventually lead to a homeomorphism. Exploring an entire dead branch may be costly: its size is exponential\footnote{due to the NP-completeness of node disjoint subgraph homeomorphism} thus exploration costs an exponential amount of time. A solution to this is to implement methods of early detection of such dead branches, i.e. pruning methods. These methods can, by nature not detect every dead branch efficiently\footnote{again, because of the NP-completeness of node disjoint subgraph homeomorphism.}. Therefore, there exists a tradeoff between strength, i.e. what proportion of dead branches it can detect, and performance, i.e. how the number of instructions used by pruning scales with the size of the inputs. Xiao's algorithm uses pruning as well, using specialized methods to suit their storage methods. We will implement different methods for RTSH, each with different tradeoffs. Although some of these pruning methods use reachability requirements, be reminded that our algorithm does not precompute paths. In those cases, reachability is calculated during runtime, optionally cached.

The input of these pruning methods comes down to an assignment of domains (sets of target graph vertices) to variables (source graph vertices). These domains represent the target graph vertex candidates for each source graph vertex in vertex-on-vertex matching. See Figure \ref{fig:simplePruningExample} for an example. Source graph vertices that are already in the partial matching have a domain of size one: the target graph vertex they have been matched to. The other vertices have domains that can be calculated in different ways (see Section \ref{sec:domain-filtering}). The pruning method then decides whether the algorithm should continue the search in this branch (i.e. do not prune) or whether the algorithm should backtrack. The different methods of deciding this are elaborated in Section \ref{sec:pruningmethods}.

\begin{figure}
\centering
\includegraphics[scale=0.6]{images/contraction/pruningExampleSimple.png}
\caption{An example case. With an empty partial mapping, the domain of $s_1$ is $\{t_1, t_2\}$, every compatible vertex in the target graph. The domain of $s_2$ is initially $\{t_5, t_6\}$ and the domain of $s_3$ $\{t_3, t_4\}$. This holds for any filtering method mentioned in this chapter. When pairs are added to the partial mapping, the domain of each source graph vertex may be reduced.}
\label{fig:simplePruningExample}
\end{figure}


\section{Domain filtering}
\label{sec:domain-filtering}
The goal of domain filtering during a search is to assign a domain of target graph vertices to each source graph vertex such that the following three criteria are satisfied:

\begin{enumerate}
\item \label{li:complete}If at least one homeomorphism can be found from this search branch, then for some homeomorphism that can be found from this search branch with vertex-on-vertex matching $M_V$ it holds that $\forall (s \to t) \in M_V . t \in \mathit{domain}(s)$.
\item \label{li:falsepositives}Each domain assignment contains as few `false positives' as possible, where a false positive is a pair of a source vertex $s$ and a target vertex $t$ such that $t$ is in the domain of variable $s$ and no homeomorphism exists from the current search path in which $s$ is matched to $t$.
\item \label{li:complexity} The process of domain filtering has a computational complexity that is as low as possible.
\end{enumerate}

In this section, we will explore different methods to obtain these domains, each with different tradeoffs between strength (performing better at criterium \ref{li:falsepositives}) and performance (performing better at criterium \ref{li:complexity}). While technically any combination of these methods can be in combination (i.e. by intersecting the domains they find), we limit ourselves with the assumption that each method is combined with every computationally cheaper filtering method.
\subsection{Labels and neighbours}
The weakest and fastest method to obtain domains for some source graph vertex $u$ is by selecting each target graph vertex $v$ that is not already used in the current partial mapping, has compatible labels and has compatible in- and outdegrees:

\begin{minipage}{\textwidth}
\begin{defn}[compability under label constraint] If $M$ is the current partial matching, then:

\begin{center}
\bgroup
\def\arraystretch{1.2}
\setlength\tabcolsep{5pt}
\begin{tabular}{lll}
\centering
$\mathit{compatible}_{\mathit{LABEL}}(s,t) := $&$t \not \in M$&$\land$\\
&$L(s) \subseteq L(t)$&$\land$\\
&$|\mathit{pred}(t)| \geq |\mathit{pred}(s)|$&$\land$\\
&$|\mathit{succ}(t)| \geq |\mathit{succ}(s)|$&\\
\end{tabular}
\egroup
\end{center}

\end{defn}
\end{minipage}

This method satisfies criterium 1 since every homeomorphism needs each source graph vertex $s$ to be matched with a target graph vertex $t$ that has at least the same label set as $s$, and the possibility of connecting with each mapped predecessor and successor of $s$. It has a low average computational complexity per source-target pair ($O(|L_S|)$).

\subsection{Free neighbours}
A somewhat stronger and somewhat slower method to obtain domains in addition to label filtering is to compare the indegree- and outdegree of each unmatched source graph vertex $s$ to other unmatched source graph vertices to the in- and outdegree of the target graph vertex $t$ to other unmatched target graph vertices. The target graph vertex must have a larger or equal indegree and outdegree compared to the source graph vertex:


\begin{minipage}{\textwidth}
\begin{defn}[compatibility under free neighbours constraint] If $M$ is the current partial matching, then:


\begin{center}
\bgroup
\def\arraystretch{1.2}
\setlength\tabcolsep{5pt}
\begin{tabular}{lll}
\centering
$\mathit{compatible}_{\mathit{FN}}(s,t) := $&$\mathit{compatible}_{\mathit{LABEL}}(s,t)$&$\land$\\
&$ |\mathit{pred}(t) \setminus M| \geq |\mathit{pred}(s) \setminus M|$&$\land$\\
&$ |\mathit{succ}(t) \setminus M| \geq |\mathit{succ}(s) \setminus M|$
\end{tabular}
\egroup
\end{center}
 
\end{defn}
\end{minipage}

This method satisfies criterium \ref{li:complete} since each unmatched neighbour must yet be matched with a target graph vertex that is (vertex disjointly) connected with the target graph vertex. For this purpose, an unused connection to the vertex is required. It has a slightly higher average computational complexity per source-target pair since (if no cached approach is chosen) each neighbour of $s$ and of $t$ needs to be counted: $O(|L_S| + |E_S|/|V_S| + |E_T|/|V_T|)$.


\subsection{Reachability of matched vertices (M-filtering)}
A strong (and computationally very expensive) method of filtering the domain we will use to calculate whether a target vertex $t$ is in the domain of a source graph vertex $s$ in addition to label- indegree and outdegree compatibility is to check that $t$ can reach the mapped vertices of successors of $s$, and that $t$ can be reached from the mapped vertices of predecessors of $s$, i.e. that candidate paths exist for the upcoming edge-path matching attempts. This pruning method does not check whether a set of paths exist that is vertex disjoint since that is the task of the main algorithm: it merely checks the existence of paths. If no path exists, then no vertex disjoint path exists, and pruning is required. An example of a domain filtered by reachability can be found in Figure \ref{fig:reachability-filtered}. Formally:

\begin{minipage}{\textwidth}
\begin{defn}[compatibility under reachability constraint]

If $M$ is the current partial mapping and $P$ is the set of paths in the target graph, then:

\bgroup
\def\arraystretch{1.5}
\setlength\tabcolsep{0pt}
\begin{tabular}{lll}
\centering
$\mathit{compatible}_{\mathit{M-REACH}}(s,t) := $&$\mathit{compatible}_{\mathit{FN}}(s,t)$&$\land$\\
&$ \mathit{compatible}_{\mathit{M-REACH, pred}}(s,t)$&$\land$\\
&$ \mathit{compatible}_{\mathit{M-REACH, succ}}(s,t)$&
\end{tabular}
\egroup

\vspace{10pt}

$$\text{where:}$$

\vspace{10pt}

\begin{tabular}{lll}
\centering
$\mathit{compatible}_{\mathit{M-REACH, pred}}(s,t) := \forall s' \in (\mathit{pred}(s) \cap M) .  \exists p \in P .$&$\mathit{first}(p)=M(s')$&$\land$\\
&$\mathit{last}(p)=t$&$\land$\\
&\multicolumn{2}{l}{$\mathit{intermediate}(p) \cap M=\emptyset$}
\end{tabular}

\begin{tabular}{lll}
\centering
$\mathit{compatible}_{\mathit{M-REACH, succ}}(s,t) := \forall s' \in (\mathit{succ}(s) \cap M) .  \exists p \in P .$&$\mathit{first}(p)=t$&$\land$\\
&$\mathit{last}(p)=M(s')$&$\land$\\
&\multicolumn{2}{l}{$\mathit{intermediate}(p) \cap M=\emptyset$}
\end{tabular}
\end{defn}
\end{minipage}

If Dijkstra's algorithm is used to find $p$, we have a complexity of $O(|E_T|+|V_T|*\mathit{log}(|V_T|))$ for each neighbour, resulting in a total average case complexity per source-target pair of $O(|L_S| + (|E_T|+|V_T|*\mathit{log}(|V_T|))*|E_S|/|V_S|)$

%pruningExampleSimple.png


\begin{figure}
\centering

\includegraphics[scale=0.6]{images/contraction/pruningExample.png}

\caption{After vertex placements $s_1 \to t_4$ and $s_2 \to t_2$, the domain for vertex $s_3$ would normally be $\{t_1, t_3\}$. However, by checking for reachability from $t_4$ we can reduce this to $\{t_3\}$. The numbers represent the matching order, and the circle styles represent labels.}
\label{fig:reachability-filtered}
\end{figure}
\subsection{Reachability of neighbourhood \hspace{12pt}(N-filtering)}
In addition to filtering domains based on reachability from- and to matched vertices that have a domain of one target graph vertex, we can also perform this on unmatched vertices that have multiple vertices in their domain. In this case, any single vertex in the domain of the source graph neighbour qualifies. If some source graph vertex $s_1$ with domain $D_1$ has an edge to some vertex $s_2$ with domain $D_2$, then for each target graph vertex $t_1 \in D_1$ there must exist a path to some vertex $t_2 \in D_2$. If not, then $t_1$ may be removed from $D_1$. Since this process is dependent on the domain of other vertices and reduces the size of domains itself, this process yields new (smaller) domains and is therefore repeated until a fixed point is reached.

\begin{minipage}{\textwidth}
\begin{defn}[compatibility under neighbourhood reachability constraint]

If $M$ is the current partial mapping and $P$ is the set of paths in the target graph, then neighbourhood reachability compatibility is defined as $$\mathit{compatible}_{\mathit{N-REACH}}(s,t):= \lim_{i\to +\infty} \mathit{nreach}_i(s, t) $$

$$\text{where:}$$

$$\mathit{nreach}_i(s, t) = \begin{cases}
                \mathit{compatible}_{\mathit{M-REACH}}(s,t)   & i = 0\\
               \mathit{ncomp}_{i, \mathit{pred}}(s, t) \land \mathit{ncomp}_{i, \mathit{succ}}(s, t) & \text{otherwise}
           \end{cases}$$

\vspace{10pt}

\begin{tabular}{lll}
\centering
$\mathit{ncomp}_{i, \mathit{pred}}(s, t) := $&\multicolumn{2}{l}{$\forall s' \in \mathit{pred}(s) . \exists t' \in V_T . \mathit{nreach}_{i-1}(s',t') \land $}\\
&$\exists p \in P .$  & $\mathit{first}(p)=t' \land$ \\
&&$\mathit{last}(p)=t \land$ \\
&&$\mathit{intermediate}(p) \cap M = \emptyset$
\end{tabular}

\begin{tabular}{lll}

$\mathit{ncomp}_{i, \mathit{succ}}(s, t) := $&\multicolumn{2}{l}{$\forall s' \in \mathit{succ}(s) . \exists t' \in V_T . \mathit{nreach}_{i-1}(s',t') \land $}\\
&$\exists p \in P .$  & $\mathit{first}(p)=t \land$ \\
&&$\mathit{last}(p)=t' \land$ \\
&&$\mathit{intermediate}(p) \cap M = \emptyset$
\end{tabular}

\end{defn}
\end{minipage}

\section{Pruning methods}
\label{sec:pruningmethods}
\subsection{ZeroDomain pruning}
\label{sec:zerodomainpruning}
The ZeroDomain pruning method decides to backtrack if and only if one of the domains is empty, i.e. there exists some source graph vertex that cannot be matched with any target graph vertex. Since each source graph vertex must be matched with some target graph vertex in a subgraph homeomorphism, this implies the search is in a dead branch. A homeomorphism cannot be found in the current search branch as no potential match exists for this source graph vertex. An example of a domain assignment where ZeroDomain prunes the search space is shown in Table \ref{tab:zerodomain}.

\label{sec:emptyDomain}
\begin{table}
\centering
\begin{tabular}{|l|l|}
\hline
\textbf{Source graph vertex} & \textbf{Target graph candidates} \\ \hline
$s_1$                          & $\{t_1, t_3, t_5\}$         \\ \hline
$s_2$                          & $\{t_1, t_2, t_3, t_4\}$       \\ \hline
$s_3$                          & $\{t_2, t_3, t_6\}$      \\ \hline
$s_4$                          & $\emptyset$                      \\ \hline
\end{tabular}
\caption{If the empty domain pruning method detects an empty target graph domain for some source graph vertex, backtracking is initiated. This is the case if the possible target graph candidates are as shown as in this table.}
\label{tab:zerodomain}
\end{table}

\subsection{AllDifferent pruning}
\label{sec:alldifferent}
The AllDifferent constraint specifies that given some assignment of domains to variables, each variable should have a non-empty domain (i.e. AllDifferent is stronger than ZeroDomain) \textit{and} and some injective mapping exists from variables to values in their domains. Since a homeomorphism requires the vertex-vertex mapping to be injective (and extending a non-injective vertex-vertex mapping can never make it injective again), the AllDifferent pruning algorithm backtracks whenever no such injective mapping exists. Alldifferent uses quadratic space since each domain needs to be known at the same time (in contrast with zero-domain, in which the knowledge of each domain separately is sufficient). An example of a domain assignment where AllDifferent prunes the search space but ZeroDomain does not is shown in Table \ref{tab:alldifferent}. 

\begin{table}
\centering
\begin{tabular}{|l|l|}
\hline
\textbf{Source graph vertex} & \textbf{Target graph candidates} \\ \hline
$s_1$                          & $\{t_1, t_2, t_3\}$      \\ \hline
$s_2$                          & $\{t_1, t_2\}$              \\ \hline
$s_3$                          & $\{t_2, t_3\}$              \\ \hline
$s_4$                          & $\{t_1, t_3\}$             \\ \hline
\end{tabular}
\caption{In this example, four source graph vertices have a total domain of only three target graph candidates. By the pigeonhole principle, no injective assignment is possible. AllDifferent recognises this and initiates backtracking.}
\label{tab:alldifferent}
\end{table}




\section{When to apply}
When a pruning method has been selected and a strategy to obtain the domains, one must lastly decide when to apply the pruning method.
\subsection{Runtime calculation}
The simplest option is to calculate the domains and run the pruning strategy each time a vertex is selected for usage in a matching. This could be a target graph vertex that is selected as for vertex-on-vertex matching or a target graph vertex that is part of a path in edge-on-path matching. This changes the partial matching and thus the domains. The pruning method then decides to allow it or to disallow it.
\subsection{Caching domains - incremental domain calculation}
Another option is to cache the domain of each variable and update it based on the current partial matching. This saves valuable time but requires quadratic space\footnote{i.e. some constant $c$ exists such that the space required has an upper bound of $c * |V_{target}|^2$} (for each source graph vertex $\in |V_{source}|$ it needs to store a domain of average size $O(|V_{target}|)$). If M-reachability filtering or N-reachability filtering is used, paths need to be cached as well. That is, for each edge $\in E_{source}$ we need to store a path of $O(|V_{target}|)$ vertices.
\subsection{Parallel calculation}
Finally, we can perform the domain filtering and procedure in a separate CPU thread while the algorithm continues without backtracking. The pruning thread queries the current matching and calculates whether pruning is appropriate for that matching. If it decides that it is not, it re-queries the current matching. Otherwise, it will interrupt the main thread and signal it to backtrack until the pruning method does not detect a dead branch anymore. This method cannot perform caching, since that requires updating the cache \textit{every} time the partial matching is extended - and performance benefits are only gained when the partial matching is only occasionally queried.

\section{Proof: contraction preserves subgraph homeomorphism}
\label{proof:contractionHomeo}
\begin{proof}

Let $\mathit{vcontract}_{S' \to S}$ and $\mathit{econtract}_{S' \to S}$ be the (injective) vertex-to-vertex mapping and (injective) edge-to-path mapping from the contracted graph $S_{cont}$ to $S$, respectively, and let $(\mathit{vmap}_{S \to T}, \mathit{emap}_{S \to T})$ be some vertex disjoint subgraph homeomorphism from $S$ to $T$, respectively.

We then construct a vertex-to-vertex mapping $\mathit{vmap}_{S' \to T}$ by mapping each vertex $v \in V_{S'}$ to $\mathit{vmap}_{S \to T}(\mathit{vcontract}_{S' \to S}(v))$. Recall from the definition of subgraph homeomorphism that $\forall s \in V_S . L_S(s) \subseteq L_T(\mathit{vmap}_{S to T}(s))$. This holds for $S$ and $T$ since we assumed a subgraph homeomorphism existed between the two.

Since for every vertex $v \in V_{S'}$ we have $\mathit{vcontract}_{S' \to S}(v) \in V_s$, it also holds that $\forall s' \in V_{S'} . L_S(\mathit{vcontract}_{S' \to S}(s')) \subseteq L_T(\mathit{vmap}_{S \to T}(\mathit{vcontract}_{S' \to S}(s')))$. Moreover, since labels of remaining vertices are preserved through contraction, we have $\forall v \in V_{S'}.L_{S'}(v)=L_S(\mathit{vcontract}_{S' \to S}(v))$. Therefore, it holds that: $\forall s' \in V_{S'} . L_{S'}(s') \subseteq L_T(\mathit{vmap}_{S \to T}(\mathit{vcontract}_{S' \to S}(s')))$. Therefore, this vertex-to-vertex mapping satisfies the first prequisite out of three for vertex disjoint subgraph homeomorphism.

We then construct an edge-to-path mapping $\mathit{emap}_{S' \to T}$. For each edge $(u, v) \in E_{S'}$, we take the edges of the path $\mathit{econtract}_{S' \to S}((u, v))$ and concatenate the edges of the corresponding paths in $T$ to obtain a new path $p'$. We then add $((u, v), p')$ to $\mathit{emap}_{S' \to T}$. Recall from the definition of subgraph homeomorphism that:

\begin{tabular}{ll}
$\forall(s_1, s_2)\in E_S .$&$ \mathit{first}(\mathit{emap}_{S \to T}(s_1, s_2))=\mathit{vmap}_{S \to T}(s_1)\land$\\
&$\mathit{last}(\mathit{emap}_{S \to T}(s_1, s_2))=\mathit{vmap}_{S \to T}(s_2)$.
\end{tabular}

Applying this formula to two consecutive edges allow us to conclude:

\begin{tabular}{ll}
$\forall(s_1, s_2), (s_2, s_3)\in E_S . $&$\mathit{first}(\mathit{emap}_{S \to T}(s_1, s_2))=\mathit{vmap}_{S \to T}(s_1)\land$\\
&$\mathit{last}(\mathit{emap}_{S \to T}(s_2, s_3))=\mathit{vmap}_{S \to T}(s_3)$. 
\end{tabular}

And similarly for more concatenated edges. Choosing the edge sequences such that each edge sequence in $S$ corresponds to a single edge in $S'$ gives us:

\begin{tabular}{ll}
$\forall(s'_1, s'_2)\in E_{S'} . $&$\mathit{first}(\mathit{emap}_{S \to T}(\mathit{firstEdge}(\mathit{econtract}(s'_1, s'_2))))=\mathit{vmap}_{S \to T}(\mathit{vcontract}(s_1))\land$\\
&$\mathit{last}(\mathit{emap}_{S \to T}(\mathit{lastEdge}(\mathit{econtract}(s'_1, s'_2))))=\mathit{vmap}_{S \to T}(\mathit{vcontract}(s'_2))$
\end{tabular}

which shows we satisfy the second prerequisite.

Lastly, we prove that performing a single contraction does not violate the third prerequisite

$$\forall p \in \mathit{values}(emap_{S \to T}) . \forall x \in \mathit{intermediate}(p) . \not \exists p' \in (\mathit{values}(\mathit{emap}_{S \to T}) \setminus \{p\}) . x \in \mathit{intermediate}(p')$$

and induce that performing any number of contractions does not. Obviously, the base case holds since it is our assumption (i.e. $S$ is vertex disjoint subgraph homeomorphic to $T$).

Let $s\in V_S$ have indegree 1 and outdegree 1, qualifying it for contraction. Let the edges be $(\mathit{prec}(s), s)$ and $(s, \mathit{succ}(s))$ where $\mathit{prec}(s)$ may be equal to $\mathit{succ}(s)$. We know that $\mathit{emap}_{S \to T}(\mathit{prec}(s), s)$ is internally vertex disjoint from all other paths in $\mathit{values}(\mathit{emap}_{S \to T})$, as well as $\mathit{emap}_{S \to T}(u, \mathit{succ}(u))$. They share at least an end vertex $\mathit{vmap}_{S \to T}(s)$ and possibly $\mathit{vmap}_{S \to T}(\mathit{prec}(s))$ if $\mathit{prec}(s)=\mathit{succ}(s)$. We know that $\mathit{vmap}_{S \to T}(s)$ is not the end vertex of another path since $s$ has indegree- and outdegree 1, and not the intermediate vertex of another path since that is not permitted for subgraph homeomorphism. All in all, the combined paths contain intermediate vertices $\mathit{intermediate}(\mathit{emap}_{S \to T}(\mathit{prec}(s), s))\cup\mathit{intermediate}(\mathit{emap}_{S \to T}(s, \mathit{succ}(s)))\cup\{\mathit{vmap}_{S \to T}(s)\}$ that are not intermediate vertices of other paths in $\mathit{values}(\mathit{emap}_{S \to T})$.

When we contract this vertex, we get a new edge $(\mathit{prec}(s), \mathit{succ}(s))$ with start vertex $\mathit{vmap}_{S \to T}(\mathit{prec}(s))$, end vertex $\mathit{vmap}_{S \to T}(\mathit{succ}(u))$, and as intermediate vertices \newline$\mathit{intermediate}(\mathit{emap}_{S \to T}(\mathit{prec}(s), s)) \cup \mathit{intermediate}(\mathit{emap}_{S \to T}(s, \mathit{succ}(s))) \cup \{\mathit{vmap}_{S \to T}(s)\}$. The vertex set used in $T$ remains exactly the same, thus does still not share intermediate vertices with other paths in $\mathit{values}(\mathit{emap}_{S \to T})$.
 
By induction, the third prerequisite holds. Since all prerequisites hold, $S'$ is a vertex disjoint subgraph homeomorphism of $T$.


%
%Then, $x$ is part of the middle section of at least two different paths in $T$. Let us call these paths $p_1$ and $p_2$. These paths were concatenated from \textit{vertex disjoint} paths in $\mathit{values}(M_{E}^{S to T})$. This implies that ${M_{E}^{S to T}}^{-1}(x)$ (let us call this $x'$) must be a contracted vertex. Therefore, there exists some path segment $p_1' \in \mathit{values}(M_{E}^{S to T})$ that contains $x$ as end vertex and some path segment $p_1''\in \mathit{values}(M_{E}^{S to T})$ that contains $x$ as start vertex, and similarly for $p_2'$ and $p_2''$. Then, at least one of the following is true:
%
%\begin{itemize}
%\item $x'$ has at least indegree 2 in which case we have a contradiction, since $x'$ was contracted.
%\item $x'$ has at least outdegree 2 in which case we have a contradiction, since $x'$ was contracted.
%\item $x'$ has a predecessor $prec(x')$ and successor $succ(x')$ such that $M_{E}^{S to T}(prec(x'))$
%\end{itemize}
%
%$x'$ has at least indegree 2 or outdegree 2 or $p_1$ and $p_2$ share . However, we deduced that $x'$ must have been contracted, which is not performed on vertices with these indegrees and outdegrees.
%
%Since our assumption leads to a contradiction, we have proven its negation which is $\forall p \in \mathit{values}(M_{E}^{S' to T}) . \forall x \in \mathit{intermediate}(p) . \not \exists p' \in (\mathit{values}(M_{E}^{S' to T}) \setminus \{p\}) . x \in \mathit{intermediate}(p')$ which satisfies the third prerequisite.
\end{proof} 